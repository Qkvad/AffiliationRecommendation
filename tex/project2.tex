%%%%%%%%%%%%%%%%%%%%%%%%%%%%%%%%%%%%%%%%%
% Journal Article
% LaTeX Template
% Version 1.4 (15/5/16)
%
% This template has been downloaded from:
% http://www.LaTeXTemplates.com
%
% Original author:
% Frits Wenneker (http://www.howtotex.com) with extensive modifications by
% Vel (vel@LaTeXTemplates.com)
%
% License:
% CC BY-NC-SA 3.0 (http://creativecommons.org/licenses/by-nc-sa/3.0/)
%
%%%%%%%%%%%%%%%%%%%%%%%%%%%%%%%%%%%%%%%%%

%----------------------------------------------------------------------------------------
%	PACKAGES AND OTHER DOCUMENT CONFIGURATIONS
%----------------------------------------------------------------------------------------

\documentclass[twoside,twocolumn]{article}

\usepackage[utf8]{inputenc}
\usepackage[T1]{fontenc}

\usepackage{blindtext} % Package to generate dummy text throughout this template 

\usepackage[sc]{mathpazo} % Use the Palatino font
\usepackage[T1]{fontenc} % Use 8-bit encoding that has 256 glyphs
\linespread{1.05} % Line spacing - Palatino needs more space between lines
\usepackage{microtype} % Slightly tweak font spacing for aesthetics

\usepackage[english]{babel} % Language hyphenation and typographical rules

\usepackage[hmarginratio=1:1,top=32mm,columnsep=20pt]{geometry} % Document margins
\usepackage[hang, small,labelfont=bf,up,textfont=it,up]{caption} % Custom captions under/above floats in tables or figures
\usepackage{booktabs} % Horizontal rules in tables

\usepackage{lettrine} % The lettrine is the first enlarged letter at the beginning of the text

\usepackage{enumitem} % Customized lists
\setlist[itemize]{noitemsep} % Make itemize lists more compact

\usepackage{abstract} % Allows abstract customization
\renewcommand{\abstractnamefont}{\normalfont\bfseries} % Set the "Abstract" text to bold
\renewcommand{\abstracttextfont}{\normalfont\small\itshape} % Set the abstract itself to small italic text

\usepackage{titlesec} % Allows customization of titles
\renewcommand\thesection{\Roman{section}} % Roman numerals for the sections
\renewcommand\thesubsection{\roman{subsection}} % roman numerals for subsections
\titleformat{\section}[block]{\large\scshape\centering}{\thesection.}{1em}{} % Change the look of the section titles
\titleformat{\subsection}[block]{\large}{\thesubsection.}{1em}{} % Change the look of the section titles

\usepackage{fancyhdr} % Headers and footers
\pagestyle{fancy} % All pages have headers and footers
\fancyhead{} % Blank out the default header
\fancyfoot{} % Blank out the default footer
\fancyhead[C]{Running title $\bullet$ May 2016 $\bullet$ Vol. XXI, No. 1} % Custom header text
\fancyfoot[RO,LE]{\thepage} % Custom footer text

\usepackage{titling} % Customizing the title section

\usepackage{hyperref} % For hyperlinks in the PDF

\usepackage{amsmath}

%----------------------------------------------------------------------------------------
%	TITLE SECTION
%----------------------------------------------------------------------------------------

\setlength{\droptitle}{-4\baselineskip} % Move the title up

\pretitle{\begin{center}\Huge\bfseries} % Article title formatting
\posttitle{\end{center}} % Article title closing formatting
\title{Affiliation Recommendation using Auxiliary Networks} % Article title
\author{%
\textsc{Petra Brčić, Ivan Čeh, Sandro Lovnički}\\[1ex]%\thanks{A thank you or further information} % Your name
\normalsize University of Zagreb, Faculty of Science, Department of Mathematics \\ % Your institution
%\normalsize \href{mailto:john@smith.com}{john@smith.com} % Your email address
%\and % Uncomment if 2 authors are required, duplicate these 4 lines if more
%\textsc{Jane Smith}\thanks{Corresponding author} \\[1ex] % Second author's name
%\normalsize University of Utah \\ % Second author's institution
%\normalsize \href{mailto:jane@smith.com}{jane@smith.com} % Second author's email address
}
\date{\today} % Leave empty to omit a date
\renewcommand{\maketitlehookd}{%
\begin{abstract}
\noindent Many social networks today, beside friendships, contain various groups and communities users associate with. Therefore, we can distinguish two co-existent networks; user-to-user and user-to-group connections. The goal of this work is to calculate group affiliation recomendation for each user. Implications of those calculations span beyond social networks and can be applied to a wide range of problems.  % Dummy abstract text - replace \blindtext with your abstract text
\end{abstract}
}

%----------------------------------------------------------------------------------------

\begin{document}

% Print the title
\maketitle

%----------------------------------------------------------------------------------------
%	ARTICLE CONTENTS
%----------------------------------------------------------------------------------------

\section{Introduction}

%\lettrine[nindent=0em,lines=3]{L} orem ipsum dolor sit amet, consectetur adipiscing elit.
\blindtext % Dummy text

Maecenas sed ultricies felis. Sed imperdiet dictum arcu a egestas. 
\begin{itemize}
\item Donec dolor arcu, rutrum id molestie in, viverra sed diam
\item Curabitur feugiat
\item turpis sed auctor facilisis
\item arcu eros accumsan lorem, at posuere mi diam sit amet tortor
\item Fusce fermentum, mi sit amet euismod rutrum
\item sem lorem molestie diam, iaculis aliquet sapien tortor non nisi
\item Pellentesque bibendum pretium aliquet
\end{itemize}
\blindtext % Dummy text

Text requiring further explanation\footnote{Example footnote}.

%------------------------------------------------

\section{Models}
In this section, we will establish the notation used in models that follow. \\
Notation. Let the $N_u$ be the number of users and $N_g$ be the number of groups. We define matrix $A \in \mathbb{R}^{N_u \times N_g}$ which denotes user$\times$group matrix or adjacency matrix of affiliation network and matrix $S \in \mathbb{R}^{N_u \times N_u}$ which denotes user$\times$user matrix or adjacency matrix of friendship. \\

The task is to recommend affiliations to a given user, we tackeled all the users simultaniously. The problem can be posed as a problem of ranking affilations in order of the users interest in joining them. Methods that will be introduced later solve the problem by assigning scores to various affiliations in order to rank them. \\

Consider the adjacency matrices $A$ and $S$. We assume $S$ is symmetric, if user $i$ is friend to user $j$ then user $j$ is friend to user $i$. Matrix $S$ corresponds to undirected graph among users and $\begin{bmatrix} 	0 & A\\ 	A^T & 0 \end{bmatrix}$ corresponds to undirected bipartite graph between users and groups. We will also define a graph $C$ between all users and groups, with the combined adjacency matrix  $ C= \begin{bmatrix} 	\lambda S & A\\ 	A^T & 0 \end{bmatrix}$. Parametar $\lambda$ controls the ratio of the weight of friendship to the weight of group membership. If $\lambda = 0$ we have bipartite affiliations graph. \\

Models we are introducing are Graph proximity model and Latent factors model to calculate score matrices. 



\subsection{Graph proximity model}
We start by assuming that the graph is known and the prediction of new links between nodes is going to be examined by calculating proximity. As we mentioned, the affiliation network can be modeled by a graph, so the basic idea is that there is possible link between two nodes based on the proximity between them. Proximity can be calculated as sum of number of paths  that connect them, paths of different lengths. \\
We are going to use Katz measure for calculating proximity. Katz measure is used to measure the relative degree of influence of a node in a network.
\[ Katz(S;\beta) = \sum_{n=1}^{\infty} \beta^n S^n = \beta S +\beta^2 S^2 + \beta^3 S^3 + ...  \]
We extend the Katz measure to the bipartitie graph A
\[ Katz(A;\beta) = \beta A A^T A +\beta^2 (AA^T)^2 A + ...  \]
where in the co-occurence matrix $A A^T$, two users $i$ and $j$ are considered connected if $i$ and $j$ belong to at least one same group, i.e. $(AA^T)_{i,j} > 0$. We consider paths from user $i$ to user $j$ by $AA^T$, and then user $j$ to some other user $k$ by $AA^T$ and then user $k$ to some group by $A$. Idea is that if user $i$ shares some community with $j$ it is more likely $i$ will join some community $j$ belongs to. \\
We will now expand Katz measure on the combined graph $C$
\[ Katz(C;\beta) = \beta C +\beta^2 C^2 + ...  \]
\[ Katz(C;\beta) = \beta A +\beta^2 \lambda S A +  \beta^3( \lambda^2 S^2A + A A^T A ) + ...  \]
This Katz measure generalizes the normal Katz measure by also considering some paths from user $i$ to user $j$ by matrix $S$, then user $j$ to some group by matrix $A$. And also user $i$ to user $j$ by $S$, user $j$ to user $k$ by $AA^T$ then again user $k$ to user $j$ by $A$ and finaly user $j$ to some group by $A$. The matrix given by $Katz(C;\beta)$ can be used as score matrix.
In case of higher dimension of matrices you work with, truncated Katz is preffered, but for our problems we stick to normal Katz because matrices for testing are not high dimensional. \\
We will just roughly discuss which $\beta$ and $\lambda$ we should take. In our algorithms $\beta$ is calculated as reciprocal of the maximum absolute value of eigenvalues of matrix $A$. Usually we take $\lambda = 0.2$ but the point of that parametar is to factor the significance of user to user matrix and the connections between users itself. 

\subsection{Latent factors model}

%------------------------------------------------

\section{Discussion}

\subsection{Results}


%----------------------------------------------------------------------------------------
%	REFERENCE LIST
%----------------------------------------------------------------------------------------

\begin{thebibliography}{99} % Bibliography - this is intentionally simple in this template

\bibitem[Figueredo and Wolf, 2009]{Figueredo:2009dg}
Figueredo, A.~J. and Wolf, P. S.~A. (2009).
\newblock Assortative pairing and life history strategy - a cross-cultural
  study.
\newblock {\em Human Nature}, 20:317--330.
 
\end{thebibliography}

%----------------------------------------------------------------------------------------

\end{document}
